% *********** Document name and reference:
% Title of document
\renewcommand{\ndoctitle}{NFR: Fully implicit Newton-Raphson solver} 
% Document category acronym 
\renewcommand{\ndocname}{nfr}                      
% svn dir
\renewcommand{\svndir}{svn://forum.astro.keele.ac.uk/frames/solver/NFR/DOC}  
% Contributors to this document
\renewcommand{\ndoccontribs}{FH}

\input{chap}

Document name: \ndocname \\
SVN directory:\svndir\\
%Contributors: \ndoccontribs\\

{ \textbf{Abstract:} \slshape NFR is a fully implicit Newton-Raphson
  solver with automatic sub-time steps. It has the built-in dynamic
  network capability that selects the right size opf the network as it
  is converging. Several linear algebra options are implemented.}

%\maketitle

%\section{Introduction and context}

%Here comes standard latex as you know it .... for example a reference to \citet{herwig:99a}. The bibtex references are kept in ngastro.bib in 

%Note that you can mark 
%\index{items} 
%that should go into the index.

\section{Input and run configuration}
\index{ppn\_solver.input} Initialisation of the NFR solver is governed by
the following parameter in \texttt{ppn\_solver.input}:
\begin{itemize}
\item \index{ittd} \texttt{ittd:} the maximum number of iterations before giving up and requesting a sub-time step. 
\item \index{dgrd} \texttt{dgrd:} convergence limit for largest relative correction
\item \index{grdthreshold} \texttt{grdthreshold:} threshold for applying dgrd test
\item \index{irdn} \texttt{irdn:} switch for dynamic network inside solver: \\
\begin{tabularx}{0.9\textwidth}{lX}
  1 & rdn for testing stuff \\
  2 & rdn supported and tested \\
  3 & rdn when starting with only 1-2 species different than zero. \\
\end{tabularx}
\item \index{cyminthreshold} \texttt{cyminthreshold:} this is what the solver considers to be zero, there is some issue concerning this parameter that is documented in the code (r775)
\item \index{mat\_solv\_option} \texttt{mat\_solv\_option:} select linear algebra solver for matrix inversion:
\\
\begin{tabularx}{0.9\textwidth}{lX}
  1 & Numerical Recipies ludcmp and lubksb \\
  2 & leqs \\
  3 & Intel-MKL/ACML dgesv (double precision general solver)  \\
  4 & sparse matrix solver from intel \\
  5 & sparse matrix solver using superLU (free) \\ 
\end{tabularx}
\end{itemize}

\section{Using MKL solvers}
Using the Intel MKL solvers provides substantial performance
enhancement. Even using the standard LAPACK solver from the MKL
libraries saves about a factor of almost 10 compared to the Numerical
Recipies solver or the leqs solver for large matrices. 

Below is a performance example:\\
The problem consists of evolving an initial solar abundance with a
full network for 100yrs at T9=0.5, rho=1000. PPN is using the RDN and
so network size is variable, but typcially above 900 for most cases.
(Note, that this test is not set up to show the fastest way of PPN to
do this calculation, it is just meant to demonstrate the different
behaviour for these solvers in a differential manner).
\begin{verbatim}
solver          threads         run time
------------------------------------------
Num Rec         1               15m14.074s
LEQS            1               42m52.179s
MKL LAPACK      1               1m58.738s
MKL LAPACK      2               1m41.131s
MKL LAPACK      8               1m24.900s
\end{verbatim}



In order to use the the MKL library it is recommended to use the
scripts of the MKL distribution for the correct hardware platform that
set the environment variables (just as you do for the intel fortran
compiler). For the UVic 64bit linux boxes this would be
\begin{verbatim}
source  /opt/intel/Compiler/11.1/Current/mkl/tools/environment/mklvarsem64t.sh
\end{verbatim}
in the \texttt{.bashrc} file.

On Falk's Intel Mac with 32bit intel 10 compiler this would be:
\begin{verbatim}
source /Library/Frameworks/Intel_MKL.framework/Versions/10.0.1.014/tools/e\
nvironment/mklvars32.sh
\end{verbatim}

Depending on the threading model the MKL solvers employ threading
techniques to enhance performance. These are, for example, OpenMP
instructions. In order to tell the MKL library how many cores are
available the code looks first at the environment variable

\begin{verbatim}
export OMP_NUM_THREADS=2
\end{verbatim}
then at 
\begin{verbatim}
export MKL_NUM_THREADS=2
\end{verbatim}
These have to be set in the \texttt{.bashrc} file.
There are also run time options to change the threading level.

Then, it is just a matter to uncomment the desired lines in the Make.local file and things should be working.

\section{Using ACML solvers}

For machines with AMD processors ACML libraries provide a better performance than MKL solvers, even using intel-fortran. ACML can be downloaded and it is free.
Please check frames/mppnp/CODE/Make.local.monster for an example on how to call ACML libraries instead of MKL libraries. 




\section{The use of CYMINFORMAL (or cyminthreshold), and how to use the static network}

CYMINFORMAL is the main parameter regulating the rdn, telling what flux (cyminformal $\sim$ flux$\times$dt) 
should be used and what not. Notice that the network is smart, and can handle also cases where at the first
iterations x$\_i$ = 0.d0, and its destruction fluxes are zero.
CYMINFORMAL can be changed, but you should keep in mind what problem you are solving. Ask yourself what is the 
smallest abundance that you want to resolve, and check the size of dt. Do few tests.
The default value (1d-30) is safe for whatever conditions we have tested. For sure you may use higher values.    

By setting CYMINFORMAL $\leq$ 0.d0, the rdn is automatically converted to the static network.
It is obvious if you think. Once, the static network was places for irdn=0. However, RH and FH 
did not like the zero option as indicative of something, so now this is not the case anymore.
The static network is hide in the rdn, which turn to be static once no selection is done with 
CYMINFORMAL. 




Here below it is reported a brief discussion, that explain how to use CYMINFORMAL.

FH:\\
I don’t know since when this does not work anymore, but I known that I did plots of NVAR, the number of species, as a function of radius for an AGB star model and it would select <70 species for the H-burn shell. Now, for a simple H burn at 55MK on solar abundance distribution it selects $\gtrsim$ 600. 

MP:\\
I have checked and it is fine to have that large number of species, given CYMINFORMAL = 1.d-30 in ppn\_solver.input.
CYMINFORMAL is basically the minimum flux*dzeit considered to build the dynamic network.
At 55MK, for instance the C13an flux at the first step is\\

 C  13HE  4NEUT = 5.744E-24

which is right if you consider that the rate is ~ 9e-20. This trigger neutron captures, e.g,\\

BE  7NEUT PROT = 1.994E-26\\
FE 56NEUT OOOOO= 1.377E-27\\
...\\

Notice that even by setting to zero the C13(a,n), the number of species will not decrease much, since there will be small fluxes (but $\geq$ 1.d-30) of proton captures. E.g.,\\

CA 44PROT OOOOO= 7.838E-26\\
...\\
K  40PROT NEUT = 7.363E-28\\

So, if for instance you set in the last stable ppn revision CYMINFORMAL=1.d-15, you will see that nvar1 $\sim$ 50.
The results are basically the same in this case, if you use cyminformal=1.e-15 or 1.e-30.
Differences beyond grdthreshold are e.g\\

<    22     9.  19.   1   1.90708E-10  F  19\\
---\\
>    22     9.  19.   1   1.90690E-10  F  19\\

The default CYMINFORMAL is extremely conservative, and it can be safely used whatever nucleosynthesis scenario is simulated, and whatever realistic timesteps adopted (also with fraction of a second).
On the other hand, CYMINFORMAL could/should be optimized depending on the simulations, in particular before extensive simulations like for future CCSN and SNIa runs. 

In general, it makes not much sense to use cyminformal < grdthreshold. 
Nevertheless, I have tested this in few cases and it is not so bad. 
At this point, grdhtreshold is not anymore reliable, better assume cyminformal as new grdthreshold.
Since the code does not break and produce results ok, for now I will not force to be cyminformal > grdthreshold.
However, I would not recommend to do that.




% \fig{fig:HRD1} shows that for similar overshoot ...


% --------------------------- end matter ------------------------
\vfill
\section{Document administration}

\subsection{History} 
This document history complements the svn log.

\begin{tabular*}{\textwidth}{lll}
\hline
Authors & yymmdd & Comment \\
\hline
FH & 090708 & set-up this document \\
\hline
\end{tabular*}

\subsection{TODO \& open issues} 
\begin{tabular*}{\textwidth}{lll}
\hline
Requester & yymmdd & Item \\
\hline
xx & 090909 & this needs to be added \\
\hline
\end{tabular*}


% --------------- latex template below ---------------------------



%\begin{figure}[htbp]
%   \centering
%%   \includegraphics[width=\textwidth]{layers.jpg} % 
%      \caption{}   \includegraphics[width=0.48\textwidth]{FIGURES/HRD90ms.png}  
%   \includegraphics[width=0.48\textwidth]{FIGURES/HRD150ms.png}  
%
%   \label{fig:one}
%\end{figure}
%
%\begin{equation}
%Y_a = Y_k + \sum_{i \neq k} Y_i
%\end{equation}
%
%{
%%\color{ForestGreen}
%\sffamily 
%  {\center  --------------- \hfill {\bf START: Some special text} \hfill ---------------}\\
%$Y_c$ does not contain ZZZ but we may assign one $Y_n$ to XYZ which is the decay product of the unstable nitrogen isotope JJHJ. %
%
%{\center ---------------  \hfill {\bf END:Some special text} \hfill ---------------}\\
%}

